\documentclass[a4paper,12pt]{article}
\usepackage[T2A]{fontenc}
\usepackage[english]{babel} % языковой пакет
\usepackage{graphicx} % для картинок
\usepackage{amsmath,amsfonts,amssymb} %математика
\usepackage{mathtools}
\everymath{\displaystyle}

\begin{document}
\section{Вступление...GO!}
Go - опенсорсный язык, который частично финансируется Google`ом.
\subsection{Как автор познакомился с Go?}
По работе необходимо было сохранять кучу информации автоматизированно, (crawling) - 
доставание информации с различных сайтов. (Есть лекция от 20го сентября на эту тему). Эта
штукенция должна условно парсить сайты и краулить кучу различных сайтов (чем больше тем лучше)
 и, однажды вернувшись на старый краулер не смогли разобраться. Взяв инициативу на себя он 
 с коммандой предложил начальству переписать его для возможности поддержки его на длительное время. 
 %очень интересно, да да
 \subsection{Плюсы Go}
 \begin{enumerate}
    \item Продуктивная разработка
    \item Куча пакетов, большое сообщество (правда в основном китайцы, так что хер ногу сломит, пока разберешься)
    \item Статическая типизация, проверки во время компиляции
    \item Легковесная асинхронность. Нет всего того, что есть в питоне и всякой хуйне, есть что то лучше
 \end{enumerate}
\subsection{Минусы Go}
\begin{enumerate}
    \item Garbage, Collected in One Langueage (сборщик мусора работает стремно)
    \item Тулинг
    \item Не подходит для ML
\end{enumerate}
Garbage Collector - это сборщик мусора на Go. Он запускается при некоторых триггерах и потом все чистит, что долго. 
Он срабатывает, когда весь размер кучи обьектов превышает размер живых обьектов, умноженный на 2. В чем суть - он запускается часто 
под конец работы программы (когда большая куча). Тупо долго, фу. 
\subsection{Практика}
\subsubsection{Установка Go}
\begin{enumerate}
    \item Гуглим Install Go
    \item Устанавливаем
    \item Добавляем в PATH
    \item пункт чтобы был больше список
    \item Успех
    \item Устанавливаем AIR
    \item Устанавливаем модуль GIM
\end{enumerate}
\subsubsection{Философия Go}
В Go главное простота, он был изначально был спроектирован очень простым, для того, чтобы у любого решения 
задачи было одно решение, котоорое лишь требуется найти
\subsubsection{Создаем веб сервис на Go}
Ну тут я хз что конспектить, тут как бы он просто пишет код... 

В папку cmd всегда кидается сам сервис (изначальное приложение). В папку pkg кидаем те файлы, которыми мы хотим/предпологаем деляться с остальными программистами (хз как)

В начале каждого файла Go должно быть название пакета, в котором он располагается. 

Переменные в Go создаются с помощью ключевого слова var, далее указывается название переменной. Также можно вместо var использовать :=. 

Если какая либо функция возвращает значение - можно буквально его проигнорировать, просто используя джокер переменную нижний слеш.
%Я официально нихуя не понимаю...   

\end{document}