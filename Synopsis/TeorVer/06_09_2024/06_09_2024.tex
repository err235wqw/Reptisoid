\documentclass[a4paper,12pt]{article}
\usepackage[T2A]{fontenc}
\usepackage[english]{babel} % языковой пакет
\usepackage{graphicx} % для картинок
\usepackage{amsmath,amsfonts,amssymb} %математика
\usepackage{mathtools}
\everymath{\displaystyle}

\begin{document}
\section{Условные распределения}\label{yslov_rasp}
\subsection{Функции регрессии}\label{func_regress}
Пусть $\Omega$, $\mathcal{F}$, $P$ - вероятностное пр-во 

$\Omega$ - мновжество элементарных методов эксперимента

$\mathcal{F}$ - $\sigma $ алгебра событий

$P$ - вероятностная мера; $P : \mathcal{F} \rightarrow [ 0 ; 1] $

$P(A)$ - вероятность события
\\
\\
\textbf{\textit{\underline{Определение:}}} \textit{Случайной величиной называется $\xi : \Omega \rightarrow \mathbb{R}$ является измеримой, т.е $\forall x \in \mathbb{R} \{\omega : \xi(\omega) < x\} \in \mathcal{F}$ .}\\
\\
\textit{\textbf{\underline{Определение:}} Функция распределения случайной величины $\xi$} \[\mathcal{F}_{\xi}(x)=P\{\omega:\xi(\omega)<x\}\]
\\
\\
\textit{\textbf{\underline{Определение:}} Функция распределения - это двумерный вектор (случайная величина)} \[\mathcal{F}_{\xi \eta }(x,y) = P\{\omega: \xi(\omega)<x, \eta(\omega)<y\} ,  \forall x,y \in \mathbb{R}^{2}\]
\\
\\
\textit{\textbf{Свойства:}}
\begin{enumerate}
    \item $0 \leq \mathcal{F}_{\xi \eta}(x,y) \leq 1, \forall x,y \in \mathbb{R}^{2}$
    \item $\mathcal{F}_{\xi \eta}(x_{0},y)$ - неубывающая непрерывная слева по $y$
    \item $\mathcal{F}_{\xi \eta}(x, y_{0})$ - неубывающая и непрерывная слева по $x$
    \item $\mathcal{F}_{\xi} (x) = \lim_{y \to +\infty} \mathcal{F}_{\xi \eta}(x, y)$
    \item $\mathcal{F}_{\eta} (y) = \lim_{x \to +\infty} \mathcal{F}_{\xi \eta}(x, y)$
    \item $lim_{x,y \to +\infty} \mathcal{F}_{\xi \eta}(x, y) = 1$
    \item $\lim_{y \to -\infty}\mathcal{F}_{\xi \eta}(x, y) =\lim_{y \to -\infty}\mathcal{F}_{\xi \eta}(x, y) = \lim_{x,y \to -\infty}\mathcal{F}_{\xi \eta}(x,y) = 0$
\end{enumerate}
\textit{\textbf{\underline{Определение:}} Случайные величины $\xi$ и $\eta$ - независимые, если $\mathcal{F}_{\xi \eta}(x, y) = \mathcal{F}_{\xi}(x)\cdot  \mathcal{F}_{\eta}(y)$}
\\
\\
\textit{\textbf{\underline{Определение:}} Условной вероятностью события $A \in \mathcal{F}$ при условии, что наступило $B \in \mathcal{F}, (P(B)>0) $ называется $P(A|B)= \frac{P(A \cap B)}{P(B)}$ }
\\
\\
\textit{\textbf{\underline{Определение:}} Условным респределением случайной величины $\eta$ относительно случайной величины $\xi$ называется
\[ 
\mathcal{F}_{\eta|\xi}(x,y) = \left\{ \begin{array}{l}
        \frac{\mathcal{F}_{\xi \eta}(x,y)}{\mathcal{F}_{\xi}(x)}; \mathcal{F}_{\xi}(x)>0\\
        0; \mathcal{F}_{\xi}(x)=0
        \end{array}
    \right\}
\]
}
%Меня заебал латех и Теорвер на 3ей фотке
\end{document}